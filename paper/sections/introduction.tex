% ============================================
% INTRODUCTION
% ============================================

\subsection{Motivation}

Housing markets exhibit dramatic boom-bust cycles that profoundly affect household wealth, financial stability, and macroeconomic performance. A growing body of research points to the importance of \textit{narratives}---the stories, beliefs, and viral ideas that shape how people interpret market signals \citep{shiller2017narrative}. The prevailing view suggests that narratives act as independent demand shocks: a viral ``buy now'' story triggers distinct waves of purchasing behavior.

However, the transmission mechanism from ``story'' to ``statistic'' remains poorly understood. Does a bullish narrative \textit{always} trigger a boom? Or does the physical reality of the market---inventory constraints, supply elasticity---act as a gatekeeper? Standard behavioral models often treat narratives as frictionless demand shifters, implying a universal link between attention and action. In contrast, complex systems theory suggests that such feedback loops should be highly state-dependent, emerging only when the system is under specific stress or constraint.

This paper tests these competing views by asking: \textbf{Is narrative transmission universal, or is it gated by market frictions?}

\subsection{Research Questions}

We address two interrelated questions using a rigorous measurement approach:

\begin{enumerate}[label=\textbf{RQ\arabic*:}]
    \item \textbf{Universal vs. Conditional:} Does narrative attention predict housing transaction volume on average, or is the relationship conditional on market state?
    \item \textbf{The Role of Frictions:} Do physical constraints (inventory shortages) and structural constraints (supply inelasticity) amplify or dampen the conversion of narrative attention into economic action?
\end{enumerate}

\subsection{Preview of Approach and Findings}

To measure these relationships consistently, we overcome a common measurement error in Google Trends analysis by implementing a \textbf{pooled within-keyword z-standardization} procedure that makes search intensity comparable across 127 U.S. Designated Market Areas (DMAs). We further aggregate all housing outcomes to the DMA level to ensure precise identification alignment.

Our findings challenge the simple ``narrative-driven'' view and support a ``friction-gated'' model:

\begin{enumerate}
    \item \textbf{Null Average Effect:} In the full panel (2012--2024), the average predictive effect of narrative attention on transaction volume is statistically indistinguishable from zero ($p=0.78$). This suggests that in ``normal'' times, high search volume is often noise rather than signal.
    
    \item \textbf{The Friction Gate:} We document profound state-dependence. Narrative attention becomes a significant predictor of volume \textit{only} in high-friction regimes:
    \begin{itemize}
        \item \textbf{State Friction (Inventory):} When inventory is low (tight market), ``Buy'' narratives significantly predict higher transaction volume ($\beta > 0, p=0.03$).
        \item \textbf{Structural Friction (Elasticity):} In supply-inelastic markets, the narrative-volume link is directionally stronger ($p=0.095$).
    \end{itemize}
    
    \item \textbf{Conditional Amplification:} We find that frictions act as a selective conductor. In tight markets, ``Buy'' narratives are significantly amplified. While we do not find significant interactions for ``Risk'' narratives, the dominance of bullish sentiment suggests that physical shortages may override bearish sentiment, forcing valid demand to compete aggressively (amplifying FOMO).
\end{enumerate}

\subsection{Contributions}

This paper makes three primary contributions:

\paragraph{Contribution 1: Rigorous Identification.} We show that previous findings of ``universal'' narrative effects may be artifacts of measurement error (non-comparable scales) and unit mismatch. By correcting these measurement issues via pooled within-keyword z-standardization and DMA aggregation, we establish a robust null baseline for the average effect.

\paragraph{Contribution 2: The ``Friction Gate'' Theory.} We provide empirical evidence that frictions are not merely impediments to efficiency but are \textit{conductors} of narrative transmission. We show that narratives require a ``stressed'' system (low inventory) to crystallize into observable economic fluctuations.

\paragraph{Contribution 3: Complex Systems Evidence.} Our findings align with a complex adaptive systems (CAS) view, where macro-level patterns (booms) emerge from micro-level interactions only under specific phase-transition conditions (scarcity), linking narrative economics with the physics of phase transitions.

\subsection{Paper Structure}

The remainder of this paper is organized as follows. Section \ref{sec:literature} reviews the relevant literature. Section \ref{sec:data} describes our standardized data pipeline. Section \ref{sec:results} presents the main results, reconciling the null average effect with strong conditional significance. Section \ref{sec:discussion} interprets the "Friction Gate" mechanism. Section \ref{sec:conclusion} concludes.
