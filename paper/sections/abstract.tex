% ============================================
% ABSTRACT
% ============================================

\begin{abstract}
We examine the relationship between narrative attention and housing market dynamics using a novel identification strategy that corrects for cross-market comparability in Google Trends data. In a rigorously constructed panel of 127 U.S. Designated Market Areas (DMAs) from 2012--2024, we find that the average predictive effect of narrative attention on transaction volume is statistically indistinguishable from zero ($p=0.78$), challenging theories of unconditional narrative-driven cycles. However, this null result masks profound state-dependent heterogeneity. We document a robust ``Friction Gate'' mechanism: narrative attention significantly predicts transaction volume only in high-friction regimes---specifically when inventory is low ($p=0.03$) and when housing supply is structurally inelastic ($p=0.095$). In these constrained environments, ``Buy'' narratives show significantly stronger predictive power for transaction volume, while ``Risk'' narratives do not exhibit significant state-dependent interactions. These findings suggest that physical scarcity acts as a selective conductor for bullish narrative signals, and that structural and state-dependent frictions are necessary conditions for narratives to have predictive relevance for market activity.
\end{abstract}
