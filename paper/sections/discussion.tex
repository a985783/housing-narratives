% ============================================
% DISCUSSION
% ============================================

This section interprets our empirical findings through the lens of the "Friction Gate" framework, discusses policy implications, and acknowledges limitations.

\subsection{The "Friction Gate" Mechanism}

Our central finding---that narrative attention predicts transaction volume only in high-friction regimes---fundamentally refines the understanding of narrative economics. Prevailing theories often treat narratives as independent demand shocks that ubiquitously influence behavior. Our results suggest a more nuanced, "gated" transmission mechanism.

\paragraph{Friction as Conductor.} In a frictionless market with abundant inventory, an increase in "Buy" search intensity does not necessarily translate into aggregate volume growth. Potential buyers may search, browse, but ultimately not transact, or their demand may be absorbed smoothly without creating a localized "frenzy." However, when inventory is low (a physical constraint), the system becomes "critical." In this state, the same increase in search intensity triggers a statistically significant behavioral response, where the scarcity itself validates and amplifies the narrative signal.

\paragraph{Differential Amplification.} Our results show that scarcity primarily amplifies the predictive power of "Buy" narratives. While we do not find significant interactions for "Risk" narratives (recession fears) in our sample, this may reflect the specific inventory-constrained regime of the 2012--2024 period. The results suggest that frictions gate transmission based on market necessity. In a tight market, the "need to buy" appears to dominate the narrative landscape, forcing valid demand to compete aggressively.

\paragraph{Quantity Rigidity.} This gating effect implies a form of \textit{downside quantity rigidity}. When inventory is constrained, positive sentiment (Buy) can only express itself through price competition or rapid turnover, as quantity cannot expand. The lack of significant "Risk" interactions suggests that in these regimes, bearish sentiment may be less influential than the physical reality of scarcity.

\subsection{Implications for Narrative Economics}

We propose that "Friction" should be considered a core variable in narrative models.
\begin{enumerate}
    \item \textbf{State Dependence:} The coefficient of narrative transmission ($\beta$) is not a constant; it is a function of the system's slack ($\beta(Inventory)$).
    \item \textbf{Measurement:} Future studies must account for market tightness. Estimating average effects across heterogeneous regimes is likely to yield null results, masking the true underlying dynamics.
\end{enumerate}

\subsection{Implications for Policy}

For macroprudential policymakers, our findings imply that monitoring "Viral Attention" alone is insufficient. The danger zone is the intersection of \textbf{High Attention} and \textbf{Low Inventory}. This is the specific quadrant where "stories become statistics" and where feedback loops are most likely to take hold. Policy interventions (e.g., cooling measures) might be most effective when targeted at these specific regime intersections rather than applied broadly.

\subsection{Limitations}

\subsubsection{Observation Period}
Our sample (2012--2024) is dominated by the post-GFC recovery and the pandemic boom, generally characterized by tightening inventory. While we exploit cross-sectional and temporal variation in inventory, a longer panel including a true "glut" era would further validate the mechanism.

\subsection{Conclusion}

We conclude that narratives are powerful, but they are not unconditionally so. They require the "conductor" of physical friction to transmit their signal into the real economy. This "Friction Gate" theory reconciles the volatility of public attention with the rigidities of the physical market, offering a grounded path forward for complex economic modeling.
