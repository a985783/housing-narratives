% ============================================
% EMPIRICAL STRATEGY
% ============================================

This section presents our empirical strategy for testing the predictions derived from the ECR-CAS framework. We estimate a baseline predictive regression, a supply-constraint interaction model, and a standardized-outcome mechanism test, and we compare volume versus price responses.

\subsection{Baseline Specification: Narrative Prediction and Controls}

Our main regression equation is:
\begin{equation}
\label{eq:main_reg}
\begin{aligned}
\Delta \ln \text{Volume}_{m,t} = & \ \alpha_m + \delta_t + \beta_1 \Nbuy_{m,t-1} + \beta_2 \Nrisk_{m,t-1} \\
& + X'_{m,t} \gamma + \varepsilon_{m,t}
\end{aligned}
\end{equation}

where:
\begin{itemize}
    \item $\Delta \ln \text{Volume}_{m,t}$: Log change in transaction volume for DMA $m$ in quarter $t$
    \item $\alpha_m$: DMA fixed effects (absorb time-invariant DMA characteristics)
    \item $\delta_t$: Quarter fixed effects (absorb aggregate shocks: interest rates, national sentiment)
    \item $\Nbuy_{m,t-1}$, $\Nrisk_{m,t-1}$: Lagged narrative indices
    \item $X_{m,t}$: Control variables (lagged volume growth, inventory, and macro controls where available)
    \item $\varepsilon_{m,t}$: Error term
\end{itemize}

\paragraph{Expected Signs.} The sign of $\beta_1$ is ambiguous ex ante. In the post-2012 period with tight inventory, we expect a ``frustrated demand'' pattern in which higher buy-side attention predicts \textit{lower} subsequent volume. We expect $\beta_2 < 0$ for risk narratives.

\paragraph{Identification.} With DMA and time fixed effects, identification comes from \textit{within-DMA, within-quarter} variation in narrative attention relative to other DMAs. The lagged structure ($t-1$ narratives predicting $t$ outcomes) addresses reverse causality concerns, though we cannot rule out omitted third factors. We interpret results as predictive relationships with mechanistic interpretation, not causal claims.

\subsection{Supply Elasticity Heterogeneity}

To test Prediction \ref{pred:supply_constraint}, we estimate:
\begin{equation}
\label{eq:saiz_interaction}
\Delta \ln \text{Volume}_{m,t} = \alpha_m + \delta_t + \beta_1 \Nbuy_{m,t-1} + \phi \left(\Nbuy_{m,t-1} \times \text{Inelastic}_m \right) + X'_{m,t}\gamma + \varepsilon_{m,t}
\end{equation}
where $\text{Inelastic}_m$ is the (negative) Saiz elasticity measure (higher values imply tighter supply constraints).

\subsection{Mechanism Validation: Standardized Outcomes}

To distinguish behavioral heterogeneity from mechanical variance compression, we re-estimate Eq. \ref{eq:saiz_interaction} using DMA-standardized volume growth $z(\Delta \ln V)$ as the dependent variable. Persistence of $\phi$ under standardization indicates a behavioral mechanism.

\subsubsection{Reverse Causality}

A concern for the baseline specification is reverse causality: high volume $\to$ media coverage $\to$ search activity. We address this through:
\begin{enumerate}
    \item \textbf{Lagged narratives}: Using $t-1$ narratives to predict $t$ outcomes
    \item \textbf{Controlling for momentum}: Including lagged volume growth
    \item \textbf{Time fixed effects}: Absorbing national-level sentiment cycles
\end{enumerate}

\subsection{Standard Errors}

We cluster standard errors at the \textbf{DMA level} to account for spatial correlation among metros that share a media market and narrative signal. All reported baseline results use DMA clustering.

\subsection{Secondary Specifications}

\subsubsection{Volume vs. Price}

To test Prediction \ref{pred:volume_leads} (volume leads price), we estimate:
\begin{equation}
\label{eq:price_comparison}
\Delta \ln P_{m,t} = \alpha_m + \delta_t + \beta_1^P \Nbuy_{m,t-1} + \beta_2^P \Nrisk_{m,t-1} + X'_{m,t} \gamma + \varepsilon_{m,t}
\end{equation}

and compare standardized effect sizes $|\hat{\beta}_1| / \text{SD}(\Delta \ln \text{Volume})$ vs. $|\hat{\beta}_1^P| / \text{SD}(\Delta \ln P)$.

\subsection{Summary of Empirical Design}

Table \ref{tab:empirical_summary} summarizes our empirical specifications and their corresponding theoretical predictions.

\begin{table}[htbp]
\centering
\caption{Summary of Empirical Specifications}
\label{tab:empirical_summary}
\footnotesize
\begin{tabular}{llll}
\toprule
\textbf{Specification} & \textbf{Key Parameter} & \textbf{Prediction} & \textbf{Interpretation} \\
\midrule
Baseline (Eq. \ref{eq:main_reg}) & $\beta_1$ & $\lessgtr 0$ & Buy narratives predict volume (frustrated demand if $<0$) \\
& $\beta_2$ & $< 0$ & Risk narratives predict volume decline \\
\addlinespace
Supply Interaction (Eq. \ref{eq:saiz_interaction}) & $\phi$ & $> 0$ & Constraints attenuate narrative effects \\
\addlinespace
Standardized DV & $\phi$ & $> 0$ & Heterogeneity is behavioral, not mechanical \\
\addlinespace
Volume vs. Price & $|\hat{\beta}^V| > |\hat{\beta}^P|$ & Yes & Volume leads price \\
\bottomrule
\end{tabular}
\end{table}
