% ============================================
% CONCLUSION
% ============================================

This paper investigates the reflexive relationship between housing narratives, structural constraints, and housing market dynamics. We develop the ECR-CAS framework to formalize how narratives interact with state-dependent boundaries to generate endogenous market fluctuations.

Using a standardized panel of 127 U.S. Designated Market Areas from 2012--2024, our analysis reveals that the average effect of narrative attention on housing volume is null. However, this null result masks a profound underlying mechanism:

\begin{enumerate}
    \item \textbf{Friction is the Conductor.} We confirm a ``Friction Gate'' theory where physical constraints are necessary conditions for narrative transmission. Narrative attention significantly predicts transaction volume \textit{only} in high-friction regimes---when inventory is low ($p=0.03$) or supply is inelastic ($p=0.095$).
    
    \item \textbf{Conditional Amplification.} In these constrained environments, viral ``Buy'' narratives significantly amplify transaction volume. While ``Risk'' narratives do not show significant state-dependent interactions in our sample, the results highlight how physical scarcity acts as a conductor for bullish narrative transmission.
\end{enumerate}

These findings contribute to narrative economics by identifying \textit{market friction} as the primary scope condition for attention-driven dynamics. Narratives do not float freely; they must be grounded in physical scarcity to move markets. Future research should prioritize this interaction, moving beyond ``average effects'' to map the non-linear topology of the attention economy.

\subsection*{Limitations}

Several limitations warrant careful consideration when interpreting our findings. First, regarding causal identification, while our two-way fixed effects specification with lagged narrative variables addresses time-invariant heterogeneity and reverse causality concerns, it cannot fully eliminate endogeneity. Unobserved local economic shocks that simultaneously drive both media coverage and housing demand may confound our estimates. The absence of truly exogenous narrative shocks---such as randomized information treatments---prevents us from claiming definitive causal identification. Instrumental variable strategies for narrative attention remain an important methodological challenge for future research.

Second, our sample composition raises representativeness concerns. By focusing on 127 Designated Market Areas, we necessarily exclude smaller metropolitan areas, rural communities, and non-metropolitan regions where housing market dynamics may differ substantially. These excluded areas often exhibit distinct institutional features, including thinner markets, different demographic compositions, and varying degrees of financial market integration. The generalizability of our findings to these contexts remains an open empirical question.

Third, the temporal scope of our analysis (2012--2024) encompasses primarily the post-Global Financial Crisis recovery period and the subsequent pandemic housing boom. This sample notably lacks observations from pronounced housing market downturns or periods of widespread negative equity. The Friction Gate mechanism may operate differently---perhaps even asymmetrically---during bust phases when inventory accumulation and distressed sales dominate market dynamics. Additionally, our reliance on Google Trends as a proxy for narrative attention, while standard in the literature, captures only online search behavior and may miss important narrative transmission channels including traditional media, social networks, and professional intermediaries.

\subsection*{Future Research Directions}

Our findings suggest several promising avenues for theoretical and empirical extension. On the theoretical front, the ECR-CAS framework presented here could be operationalized through Agent-Based Simulation (ABS) to explore emergent phenomena at the market level. Such simulations would allow researchers to explicitly model heterogeneous agents with varying attention constraints, learning rules, and network positions. Incorporating network effects and social media propagation mechanisms would be particularly valuable, as recent evidence suggests that narrative contagion follows complex network topologies that amplify or dampen transmission depending on structural properties of the social graph.

Empirically, future research should examine the moderating role of credit constraints on narrative transmission. Integrating Home Mortgage Disclosure Act (HMDA) data would enable direct testing of whether tight lending standards attenuate or amplify the Friction Gate effect. Additionally, the post-2010 regulatory environment provides a natural laboratory for evaluating the effectiveness of macroprudential policies such as the Ability-to-Repay (ATR) and Qualified Mortgage (QM) rules. Cross-national extensions would test whether the Friction Gate mechanism generalizes to institutional contexts with different mortgage market structures, land use regulations, and cultural narratives about homeownership.

From a policy applications perspective, our framework suggests the feasibility of developing real-time monitoring systems for narrative attention and market friction indicators. Such early warning systems could identify DMAs entering the ``high attention + high friction'' danger zone, enabling preemptive policy responses.
 Future research should also explore the design of state-contingent policy instruments that automatically adjust stringency based on real-time friction indicators, moving beyond the current paradigm of uniform national policies toward geographically targeted interventions.

\subsection*{Policy Implications}

Our findings carry significant implications for housing market stabilization policy. The identification of a ``high attention + low inventory'' danger zone provides a concrete target for preemptive intervention. When search intensity for housing-related narratives spikes concurrently with depleted inventory levels, policymakers should anticipate panic buying dynamics and consider targeted measures such as temporary transaction taxes, enhanced disclosure requirements, or strategic releases of public land reserves. These friction-responsive interventions would represent a departure from the current practice of applying uniform macroprudential tools regardless of local market conditions.

The state-dependent amplification of ``Buy'' narratives suggests that broad-based cooling measures may be inefficiently blunt. During high-friction periods, policies targeting buyer expectations---such as clear communication about future supply pipelines or temporary demand management---may be more effective than traditional interest rate instruments. For central banks and financial regulators, our results underscore the importance of monitoring narrative indicators alongside conventional price and volume metrics. The Federal Reserve and similar institutions should consider incorporating attention-based variables into their financial stability surveillance frameworks, particularly when assessing risks in geographically concentrated housing markets.

\subsection*{Contributions}

This paper makes three distinct contributions to the literature on narrative economics and housing market dynamics. First, we formalize the \textbf{Friction Gate theory}, which posits that market friction serves as a necessary condition for narrative transmission. This theoretical advance moves beyond the binary debate over whether narratives matter, instead identifying the specific structural conditions under which they exert measurable influence.

Second, we document a \textbf{state-dependent narrative transmission mechanism} in which bullish narratives are selectively amplified by market constraints.
 This mechanism has important implications for understanding boom-bust cycles and designing appropriately targeted stabilization policies.

Third, we provide \textbf{empirical evidence from a complex systems perspective}, demonstrating that housing markets exhibit emergent, state-dependent behaviors that cannot be reduced to simple linear relationships. By mapping the non-linear topology of the attention-friction interaction, our analysis contributes to a growing body of research applying complexity science methods to macroeconomic phenomena.

\vspace{1em}

\noindent\textbf{Data Availability Statement}: All data used in this study are publicly available. Redfin data are from the Redfin Data Center. Google Trends data are accessible via the Trends API.

\vspace{1em}

\noindent\textbf{Disclosure Statement}: The author has no conflicts of interest to disclose.
